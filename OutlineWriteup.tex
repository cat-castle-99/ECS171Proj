\documentclass[12pt]{article}

% If you're new to LaTeX, here's some short tutorials:
% https://www.overleaf.com/learn/latex/Learn_LaTeX_in_30_minutes
% https://en.wikibooks.org/wiki/LaTeX/Basics

% Formatting
\usepackage[utf8]{inputenc}
\usepackage[margin=1in]{geometry}

% Math
% https://www.overleaf.com/learn/latex/Mathematical_expressions
% https://en.wikibooks.org/wiki/LaTeX/Mathematics
\usepackage{amsmath,amsfonts,amssymb,mathtools}

% Images
% https://www.overleaf.com/learn/latex/Inserting_Images
% https://en.wikibooks.org/wiki/LaTeX/Floats,_Figures_and_Captions
\usepackage{graphicx,float}

% Tables
% https://www.overleaf.com/learn/latex/Tables
% https://en.wikibooks.org/wiki/LaTeX/Tables

% Algorithms
% https://www.overleaf.com/learn/latex/algorithms
% https://en.wikibooks.org/wiki/LaTeX/Algorithms
\usepackage{algorithmic}

% Tikz for drawing graphs and diagrams, although using other tools and 
% importing images might be easier
% This gives a pretty good tutorial: 
% http://www.tcs.uni-luebeck.de/downloads/mitarbeiter/tantau/2012-gd-presentation.pdf
\usepackage{tikz}

% Title content
\title{ECS 171 One Page Writeup}
\author{\small Arielle Soomi Yoo, Samuel Robert Perelgut, Catherina Castillo, Huy Minh Tran, Mia Ma, William T Wu }
\date{April 10, 2021}

\begin{document}

\maketitle

% In general, don't edit anything above this
% Just use this format for each homework
\section*{Group 16 Goals: }

\subsection*{}
The overall purpose of this project is to create a website that can assist 
medical professionals in diagnosing whether a patient is at risk for a heart 
disease. This problem is important because these kinds of judgments can save 
es by allowing us to predict what sort of life threatening issues may affect 
patients. This problem is interesting because it would give doctors a lot of 
extra information and help when making decisions on how to treat their 
patients. Further, a few past studies on the matter have shown that in some 
cases artificial intelligence programs have superior judgement when making 
these decisions then medical professionals do [5]. That is not to say that 
our program is superior to or a replacement for a doctor, but rather it is 
a reliable extra tool available to them that will help them ensure the best 
outcome for their patients. While this project focuses only on heart disease,
future iterations of this project conducted by us or others will more than 
likely expand on this idea to encompass a wide variety of possible ailments. 
Indeed this the first step in reimagining how we, as a society, practice 
medicine.

\section*{Dataset: }

\subsection*{}
The data for this project originally comes from The UCI Machine Learning Repository and contains data donated from four different sources:

\begin{enumerate}
    \item The Hungarian Institute of Cardiology. Budapest: Andras Janosi, M.D.,
    \item The University Hospital, Zurich, Switzerland: William Steinbrunn, M.D.,
    \item The University Hospital, Basel, Switzerland: Matthias Pfisterer, M.D., and
    \item The V.A. Medical Center, Long Beach and Cleveland Clinic Foundation:Robert Detrano, M.D., Ph.D. [1, 2].
    
\end{enumerate}

In general, the 14 variables we will focus on include general background 
information on the patient, various cardiac health measurements taken from 
the patient, and the presence or absence of heart disease [2, 4]. 
The dataset includes information on 303 consecutive patients that were 
referred to the Cleveland Clinic for coronary angiography between May 1981 
and September 1984 [2, 4]. From this data, we should be able to predict 
whether a patient is at risk of heart disease or not. In general, we will 
clean the data, removing rows with missing values, and split it into a test 
set and training set. We will download the dataset directly from the kaggle 
website, which has some more information on this UCI dataset [1].

\section*{Deliverables: }

We’re planning to create a web application for purposes of allowing a user 
to interact with our created model. The idea is that the web application 
will act as an interface where a medical professional can enter in prompted 
information such as patient’s age, sex, cholestorol and more, and our 
website will output a prediction as to whether or not the patient is at risk 
for heart disease.

We also need an outline that describes our data, problem, and solution 
in more detail than is covered here. This can and should be worked on 
concurrently with the development of the webapp and model, but ultimately 
will probably be the last thing finished, so we can add the conclusion 
derived from the finished webapp and model.

\section*{Action Plan: }
\begin{enumerate}
    \item Assign roles and responsibilities -- 4/12
    \subitem a. Roles may be fluid/subject to change
    \item Clean/format the data (EDA) -- 4/16
    \item Frontend -- 4/23
    \subitem a. Look for web development tools that work with Python 
    \subsubitem i. Ex: Flask, nodejs
    \subitem b. How to merge frontend with backend
    \item Devise algorithm -- 4/23
    \subitem a. Compare different ML strategies
    \subitem b. Data visualization
    \subitem c. Evaluation
    \item Make an API together and start coding -- 4/23
    \subitem a. Agree on inputs and outputs of functions
    \subitem b. Adjust roles if needed
    \item Testing/Test Cases -- 5/17
    \subitem a. Backend people can probably test things on their own
    \subitem b. Frontend people can test button presses/etc
    \subitem c. Connecting: might have to test together
    \item Final Outline -- 5/26
    \subitem a. Can be worked on concurrently with some of project, however should be last completed after webapp is functional to our liking
    \item Make project public on Github, and a video demo -- 5/26   
\end{enumerate}

\section*{Citations for Data}
\subsection*{}
\begin{enumerate}
    \item Main dataset: https://www.kaggle.com/ronitf/heart-disease-uci
    \item http://archive.ics.uci.edu/ml/datasets/Heart+Disease
    \item https://www.kaggle.com/ronitf/heart-disease-uci/discussion/101018
    \item Detrano, R., Janosi, A., Steinbrunn, W., Pfisterer, M., Schmid, J., 
    Sandhu, S., Guppy, K., Lee, S.,Froelicher, V. (1989). International 
    application of a new probability algorithm for the diagnosis of coronary 
    artery disease. American Journal of Cardiology, 64,304--310. 
    https://www.sciencedirect.com/science/article/pii/0002914989905249?via%3Dihub
    \item Lysaght, T., Lim, H.Y., Xafis, V. et al. AI-Assisted Decision-making in Healthcare. ABR 11, 299–314 (2019). https://doi.org/10.1007/s41649-019-00096-0
\end{enumerate}
% Don't change this
\end{document}
