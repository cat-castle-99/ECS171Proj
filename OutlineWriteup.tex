\documentclass[12pt]{article}

% If you're new to LaTeX, here's some short tutorials:
% https://www.overleaf.com/learn/latex/Learn_LaTeX_in_30_minutes
% https://en.wikibooks.org/wiki/LaTeX/Basics

% Formatting
\usepackage[utf8]{inputenc}
\usepackage[margin=1in]{geometry}
\usepackage{titlesec}
\usepackage{enumitem}

\geometry{
    a4paper,
    total={170mm,257mm},
    left=10mm,
    right=10mm,
    top=0mm,
}



% Math
% https://www.overleaf.com/learn/latex/Mathematical_expressions
% https://en.wikibooks.org/wiki/LaTeX/Mathematics
\usepackage{amsmath,amsfonts,amssymb,mathtools}

% Images
% https://www.overleaf.com/learn/latex/Inserting_Images
% https://en.wikibooks.org/wiki/LaTeX/Floats,_Figures_and_Captions
\usepackage{graphicx,float}

% Tables
% https://www.overleaf.com/learn/latex/Tables
% https://en.wikibooks.org/wiki/LaTeX/Tables

% Algorithms
% https://www.overleaf.com/learn/latex/algorithms
% https://en.wikibooks.org/wiki/LaTeX/Algorithms
\usepackage{algorithmic}

% Tikz for drawing graphs and diagrams, although using other tools and 
% importing images might be easier
% This gives a pretty good tutorial: 
% http://www.tcs.uni-luebeck.de/downloads/mitarbeiter/tantau/2012-gd-presentation.pdf
\usepackage{tikz}

% Title content
\title{\normalsize ECS 171 One Page Writeup}
\author{\small Arielle Soomi Yoo, Samuel Robert Perelgut, Catherina Castillo, Huy Minh Tran, Mia Ma, William T Wu }
\date{April 10, 2021}

\begin{document}
\titlespacing*{\section}{0pt}{0.25cm}{0.25cm}
\titlespacing*{\subsection}{0pt}{0.5cm}{0.5cm}

\maketitle

% In general, don't edit anything above this
% Just use this format for each homework
\section*{\normalsize Group 16 Goals: }

The overall purpose of this project is to create a website that can assist 
medical professionals in diagnosing whether a patient is at risk for a heart 
disease. This problem is interesting because it would give doctors a lot of 
extra information and help when making decisions on how to treat their 
patients. Further, a few past studies on the matter have shown that in some 
cases artificial intelligence programs have superior judgement when making 
these decisions then medical professionals do [2]. While this project focuses 
only on heart disease, future iterations of this project conducted by us or 
others will more than likely expand on this idea to encompass a wide variety 
of possible ailments.

\section*{\normalsize Dataset: }

The data for this project originally comes from The UCI Machine Learning 
Repository and contains data donated from four different sources.

In general, the 14 variables we will focus on include general background 
information on 303 patients, various cardiac health measurements taken 
from the patients, and the presence or absence of heart disease [1]. 
From this data, we should be able to predict whether a patient is at risk of 
heart disease or not. In general, we will clean the data, removing rows with 
missing values, and split it into a test set and training set. We will 
download the dataset directly from the kaggle website, which has some more 
information on this UCI dataset [1].

\section*{\normalsize Deliverables: }

We’re planning to create a web application for purposes of allowing a user 
to interact with our created model. The idea is that the web application 
will act as an interface where a medical professional can enter in prompted 
information such as patient’s age, sex, cholestorol and more, and our 
website will output a prediction as to whether or not the patient is at 
risk for heart disease. We also need an outline that describes our data, 
problem, and solution in more detail than is covered here. This will be 
worked on concurrently with the development of the webapp and model, but 
ultimately will be the last part finished so we can add the conclusion 
derived from the finished webapp and model.

\section*{\normalsize Action Plan: }
\begin{enumerate}[topsep=0pt,itemsep=-1ex,partopsep=1ex,parsep=1ex]
    \item Assign roles and responsibilities -- 4/12
    \item Clean/format the data (EDA) -- 4/16
    \item Frontend -- 4/23
    \item Devise algorithm -- 4/23
    \item Make an API together and start coding -- 4/23
    \item Testing/Test Cases -- 5/17
    \item Final Outline -- 5/26
    \item Make project public on Github, and a video demo -- 5/26   
\end{enumerate}

\section*{\normalsize Citations for Data}
\begin{enumerate}[topsep=0pt,itemsep=-1ex,partopsep=1ex,parsep=1ex]
    \item Main dataset: https://www.kaggle.com/ronitf/heart-disease-uci    Sandhu, S., Guppy, K., Lee, S.,Froelicher, V. (1989). International 
    \item Lysaght, T., Lim, H.Y., Xafis, V. et al. AI-Assisted Decision-making in Healthcare. ABR 11, 299–314 (2019). https://doi.org/10.1007/s41649-019-00096-0
\end{enumerate}
% Don't change this
\end{document}
